\documentclass{article}
\usepackage[margin=1in]{geometry}
\usepackage{fancyhdr}
\pagestyle{fancy}
\lhead{}
\chead{}
\rhead{Unitary Linear Transformations}
\lfoot{}
\cfoot{\thepage}
\rfoot{}
\usepackage{amsmath, amsthm, amssymb}
\usepackage{mathtools}
\usepackage{graphicx}
\usepackage{caption}
\usepackage{hyperref}
\usepackage{changepage}
\usepackage{lipsum}
\usepackage{color, colortbl} % For Coloring Table Elements
\usepackage{wrapfig}
%\usepackage{alltt}
\usepackage{listings}
\usepackage{makecell} % allows more complex table formatting
\usepackage{proof} %for inference trees
\usepackage{scrextend} % for block indents (for example)
\usepackage{cite}
\def\BibTeX{{\rm B\kern-.05em{\sc i\kern-.025em b}\kern-.08em
    T\kern-.1667em\lower.7ex\hbox{E}\kern-.125emX}}
\usepackage{url}

% for the makecell package:
\renewcommand\theadalign{bc}
\renewcommand\theadfont{\bfseries}
\renewcommand\theadgape{\Gape[4pt]}
\renewcommand\cellgape{\Gape[4pt]}

\addtolength{\hoffset}{-1cm}
\addtolength{\textwidth}{1cm}
\graphicspath{ {IMAGES/} }

\definecolor{Gray}{gray}{0.9}
\newcolumntype{g}{>{\columncolor{Gray}}c}

\title{A Unitary Linear Transformation Preserves the Inner Product of any Two Arbitrary Vectors}
\author{Warren (Hoss) Craft}
\date{Summer 2019}

\begin{document}
\maketitle
%\tableofcontents
%\newpage

\flushleft

%%%%%%%%%%%%%%%%%%%
%   Intro Parag   %
%%%%%%%%%%%%%%%%%%%

Mermin \cite{Mermin_2007} writes that,
\vspace{0.125in}
\begin{addmargin}[0.25in]{0.25in}
A linear transformation that preserves the magnitudes of all vectors is called unitary, because it follows from linearity that all magnitudes will be preserved if and only if unit vectors (vectors of magnitude 1) are taken into unit vectors. It also follows from linearity that if a linear transformation \textbf{U} is unitary then it must preserve not only the inner products of arbitrary vectors with themselves, but also the inner products of arbitrary pairs of vectors. This follows straightforwardly for two general vectors $|\phi\rangle$ and $|\psi\rangle$ from the fact that \textbf{U} preserves the magnitudes of both of them, as well as the magnitudes of the vectors $|\phi\rangle + |\psi\rangle$ and $|\phi\rangle + i|\psi\rangle$. [pg 161 (Appendix A)]
\end{addmargin}

\vspace{0.25in}

\textbf{I. A $2 \times 2$ Unitary Matrix}\par

\vspace{0.125in}
As a concrete example, consider the unitary linear transformation represented by the following $2 \times 2$ (complex) matrix \cite{wikipedia:UnitaryMatrix}:
\[
U = \begin{pmatrix}
a & b\\ -e^{i\theta}\overline{b} & e^{i\theta}\overline{a}
\end{pmatrix}
\]

where $a$ and $b$ are complex numbers such that $|a|^2 + |b|^2 = 1$, and $\overline{x}$ indicates complex conjugation of a complex number $x$.

\vspace{0.125in}

Notice that:
\begin{align*}
\det(U)      &= e^{i\theta}a \overline{a} + e^{i\theta}b \overline{b}\\
&= e^{i\theta}|a|^2 + e^{i\theta}|b|^2\\
&= e^{i\theta}(|a|^2 + |b|^2)\\
&= e^{i\theta}\\
\end{align*}
and letting $\overline{U}^{T}$ denote the conjugate transpose of $U$, we have:
\begin{align*}
U \overline{U}^{T} &=
\begin{pmatrix}
a & b\\
-e^{i\theta}\overline{b} & e^{i\theta}\overline{a}
\end{pmatrix}
\begin{pmatrix}
\overline{a} & -e^{-i\theta}b\\
\overline{b} & e^{-i\theta}a
\end{pmatrix}\\
&= \begin{pmatrix}
   a\overline{a} + b\overline{b} &
   -e^{-i\theta}ab + e^{-i\theta}ab\\
   -e^{i\theta}\overline{a}\overline{b} + e^{i\theta}\overline{a}\overline{b} &
   b\overline{b} + a\overline{a}
   \end{pmatrix}\\
&= \begin{pmatrix}
   |a|^2 + |b|^2 & 0\\
   0             & |b|^2 + |a|^2
   \end{pmatrix}\\
&= \begin{pmatrix}
   1 & 0\\
   0 & 1
   \end{pmatrix}\\
&= I
\end{align*}

Similarly we can show that $\overline{U}^{T}U = I$, and thus $\overline{U}^{T}U = U\overline{U}^{T} = I$, verifying that $U$ is unitary.

\vspace{0.25in}

\textbf{II. U Preserving the Magnitude of a Vector}\par

\vspace{0.125in}

Our unitary linear transformation represented by the matrix \textbf{U} in the previous section should preserve the magnitude of any 2D vector $x = x_0 |0\rangle + x_1 |1\rangle$ in our 2D vector space with $|0\rangle$ and $|1\rangle$ as orthonormal basis vectors. As a concrete example, note that for the unit vector $x = \frac{\sqrt{2}}{2} |0\rangle + \frac{\sqrt{2}}{2} |1\rangle = (\frac{\sqrt{2}}{2}, \frac{\sqrt{2}}{2})$ we have:

\begin{align*}
x^{\prime} &= U x\\
&=
\begin{pmatrix}
a & b\\
-e^{i\theta}\overline{b} & e^{i\theta}\overline{a}
\end{pmatrix}
\begin{pmatrix}
\sqrt{2}/2\\
\sqrt{2}/2
\end{pmatrix}\\
&= \frac{\sqrt{2}}{2}
\begin{pmatrix}
a & b\\
e^{i\theta}\overline{b} & e^{i\theta}\overline{a}
\end{pmatrix}
\begin{pmatrix}
1\\
1
\end{pmatrix}\\
&= \frac{\sqrt{2}}{2}
\begin{pmatrix}
   a + b\\
   e^{i\theta}(\overline{a} - \overline{b})
   \end{pmatrix}
\end{align*}

So, the magnitude of the transformed vector $x^{\prime}$ is:
\begin{align*}
|x^{\prime}| &= (\frac{\sqrt{2}}{2})\sqrt{|a+b|^2 + |\overline{a}-\overline{b}|^2}\\
&= (\frac{\sqrt{2}}{2})\sqrt{|a+b|^2 + |\overline{a - b}|^2}\\
&= (\frac{\sqrt{2}}{2})\sqrt{
  (a+b)\overline{(a+b)} + 
  \overline{(a - b)}(a - b)}\\
&= (\frac{\sqrt{2}}{2})\sqrt{
  (a+b)(\overline{a}+\overline{b}) + 
  (\overline{a} - \overline{b})(a - b)}\\
&= (\frac{\sqrt{2}}{2})\sqrt{
  (a\overline{a} + a\overline{b} + \overline{a}b + b\overline{b}) + 
  (a\overline{a} - a\overline{b} - \overline{a}b + b\overline{b})
  }\\
&= (\frac{\sqrt{2}}{2})\sqrt{
  2 a \overline{a} + 2 b \overline{b}
  }\\
&= \sqrt{
  a \overline{a} + b \overline{b}
  }\\
&= \sqrt{
  |a|^2 + |b|^2
  }\\
&= 1
\end{align*}

More generally, for a vector $x = x_0 |0\rangle + x_1 |1\rangle$ with magnitude $|x| = \sqrt{|x_0|^2 + |x_1|^2}$ we have:

\begin{align*}
x^{\prime}
&= U x\\
&= \begin{pmatrix}
   a & b\\
   -e^{i\theta}\overline{b} & e^{i\theta}\overline{a}
\end{pmatrix}
\begin{pmatrix} x_0\\x_1\end{pmatrix}\\
&= \begin{pmatrix}
   ax_0 + bx_1\\
   -e^{i\theta}\overline{b} x_0 +  e^{i\theta}\overline{a} x_1
\end{pmatrix}\\
&= \begin{pmatrix}
   ax_0 + bx_1\\
   e^{i\theta}(
   \overline{a} x_1 - \overline{b} x_0)
\end{pmatrix}\\
\end{align*}

So the magnitude of the transformed vector $x^{\prime}$ is:

\begin{align*}
|x^{\prime}| &=
\sqrt{|ax_0+bx_1|^2 + |e^{i\theta}(\overline{a}x_1-\overline{b}x_0)|^2}\\
&=
\sqrt{|ax_0+bx_1|^2 + |\overline{a}x_1-\overline{b}x_0|^2}
\hspace{0.25in}(\text{because }|z_1 z_2|^2 = |z_1|^2 |z_2|^2 \text{ and } |e^{i\theta}|^2 = 1) \\
&=
\sqrt{(ax_0+bx_1)\overline{(ax_0+bx_1)} + (\overline{a}x_1-\overline{b}x_0)\overline{(\overline{a}x_1-\overline{b}x_0)}}\\
&=
\sqrt{(ax_0+bx_1)(\overline{a}\;\overline{x_0}+\overline{b}\overline{x_1}) + (\overline{a}x_1-\overline{b}x_0)(a\overline{x_1} - b\overline{x_0})}\\
&=
\sqrt{
  a \overline{a} x_0\overline{x_0} +
  a \overline{b} x_0 \overline{x_1} +
  \overline{a} b \overline{x_0} x_1 +
  b \overline{b} x_1 \overline{x_1} +
  a \overline{a} x_1 \overline{x_1} -
  \overline{a} b \overline{x_0} x_1 -
  a \overline{b} x_0 \overline{x_1} +
  b \overline{b} x_0 \overline{x_0}
}\\
&=
\sqrt{
  (a \overline{a} + b \overline{b}) x_0\overline{x_0} +
  (a \overline{a} + b \overline{b}) x_1 \overline{x_1}
}\\
&=
\sqrt{
  (|a|^2 + |b|^2) |x_0|^2 +
  (|a|^2 + |b|^2) |x_1|^2
}\\
&= \sqrt{|x_0|^2 + |x_1|^2}
\hspace{0.25in}(\text{because } |a|^2 + |b|^2 = 1)
\end{align*}

Thus our unitary linear transformation \textbf{U} does indeed preserve the magnitude of an arbitrary (2D) vector (with complex coordinates).

\vspace{0.25in}

\textbf{III. U Preserving the Inner Product of Two Vectors}\par

\vspace{0.125in}

Ultimately we want to show that our unitary linear transformation \textbf{U} preserves the inner product of any two arbitrary vectors $|\phi\rangle$ and $|\psi\rangle$. That is, we want to show that:

\[
\langle U |\phi\rangle | U |\psi\rangle \rangle = \langle \phi | \psi \rangle.
\]

Let $|\phi\rangle = \beta_0 |0\rangle + \beta_1 |1\rangle$ and $|\psi\rangle = \alpha_0 |0\rangle + \alpha_1 |1\rangle$ be two arbitrary vectors in our 2D vector space with orthonormal basis $\{|0\rangle, |1\rangle\}$. Then the inner product of $\phi$ and $\psi$ is given by:

\[
\langle \phi | \psi \rangle = \sum_{i=0}^{N-1}\overline{\beta_i}\alpha_i =
\overline{\beta_0}\alpha_0 + \overline{\beta_1}\alpha_1.
\]

We have 
$U|\phi\rangle = \begin{pmatrix}
   a\beta_0 + b\beta_1\\
   e^{i\theta}(
   \overline{a} \beta_1 - \overline{b} \beta_0)
\end{pmatrix}$
and
$U|\psi\rangle = \begin{pmatrix}
   a\alpha_0 + b\alpha_1\\
   e^{i\theta}(
   \overline{a} \alpha_1 - \overline{b} \alpha_0)
\end{pmatrix}$,
so the inner product of the transformed vectors is:

\begin{align*}
    \langle U |\phi\rangle | U |\psi\rangle \rangle
    &= \langle \begin{pmatrix}
   a\beta_0 + b\beta_1\\
   e^{i\theta}(
   \overline{a} \beta_1 - \overline{b} \beta_0)
\end{pmatrix} | \begin{pmatrix}
   a\alpha_0 + b\alpha_1\\
   e^{i\theta}(
   \overline{a} \alpha_1 - \overline{b} \alpha_0)
\end{pmatrix} \rangle\\
&= \overline{(a\beta_0 + b\beta_1)}(a\alpha_0 + b\alpha_1) + 
\overline{e^{i\theta}(
   \overline{a} \beta_1 - \overline{b} \beta_0)}(e^{i\theta}(
   \overline{a} \alpha_1 - \overline{b} \alpha_0))\\
&= (\overline{a}\overline{\beta_0} + \overline{b}\;\overline{\beta_1})(a\alpha_0 + b\alpha_1) + 
e^{-i\theta}(
   a \overline{\beta_1} - b \overline{\beta_0})(e^{i\theta}(
   \overline{a} \alpha_1 - \overline{b} \alpha_0))\\
&= (\overline{a}\overline{\beta_0} + \overline{b}\;\overline{\beta_1})(a\alpha_0 + b\alpha_1) + 
(
   a \overline{\beta_1} - b \overline{\beta_0})(
   \overline{a} \alpha_1 - \overline{b} \alpha_0)\\
&= (a\overline{a}\alpha_0\overline{\beta_0} + 
\overline{a}b\alpha_1\overline{\beta_0} +
a \overline{b}\alpha_0\overline{\beta_1} +
b \overline{b} \alpha_1 \overline{\beta_1}) +
(a \overline{a} \alpha_1 \overline{\beta_1} -
a \overline{b} \alpha_0 \overline{\beta_1} -
\overline{a} b \alpha_1 \overline{\beta_0} +
b \overline{b} \alpha_0 \overline{\beta_0})\\
&= (a\overline{a}+b\overline{b})\alpha_0\overline{\beta_0} +
(a \overline{a} + b \overline{b}) \alpha_1 \overline{\beta_1}\\
&= (|a|^2 + |b|^2)\alpha_0\overline{\beta_0} +
(|a|^2 + |b|^2) \alpha_1 \overline{\beta_1}\\
&= \alpha_0\overline{\beta_0} +
   \alpha_1 \overline{\beta_1}\\
&= \overline{\beta_0} \alpha_0 +
   \overline{\beta_1} \alpha_1 \\
&= \langle \phi | \psi \rangle
\end{align*}


% \bibliographystyle{ACM-Reference-Format}
\bibliographystyle{plain}
\bibliography{bibliography.bib}

\end{document}